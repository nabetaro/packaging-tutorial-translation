\documentclass[10pt,final]{beamer}
\mode<presentation>
\usetheme{debian}
\usepackage{debiantutorial}

\hypersetup{pdftitle={Practical session 1: grep},bookmarks}
\title[Practical session 1: grep]{Practical session 1:\\ Modifying the grep package}
\author[]{Lucas Nussbaum\\{\small\texttt{lucas@debian.org}}}
\institute{\includegraphics[viewport=274 335 360 440,width=1cm]{figs/openlogo-nd.pdf}}
\date{}

\begin{document}

\frame{\titlepage}

\begin{frame}{Practical session: modifying the grep package}
\begin{enumerate}
	\item Go to \url{http://ftp.debian.org/debian/pool/main/g/grep/} and
		download version 2.6.3-3 of the package
	\item Look at the files in \texttt{debian/}.
		\begin{itemize}
			\item 		How many binary packages are generated by this source package?
			\item 		Which packaging helper does this package use?
		\end{itemize}
	\item Build the package
	\item We are now going to modify the package. Add a changelog entry and increase the version number.
	\item Now disable perl-regexp support (it is a \texttt{./configure} option)
	\item Rebuild the package
	\item Compare the original and the new package with debdiff
	\item Install the newly built package
	\item Cry if you messed up ;)
\end{enumerate}
\end{frame}

\begin{frame}{Fetching the source}
\begin{enumerate}
	\item Go to \url{http://ftp.debian.org/debian/pool/main/g/grep/} and
		download version 2.6.3-3 of the package
\end{enumerate}
\begin{itemize}
	\item Use dget to download the \texttt{.dsc} file:\\
		{\small \texttt{dget http://cdn.debian.net/debian/pool/main/g/grep/grep\_2.6.3-3.dsc}}
		\hbr
	\item According to \texttt{http://packages.qa.debian.org/grep}, \texttt{grep} version 2.6.3-3 is currently in \textsl{stable} (\textsl{squeeze}). If you have \texttt{deb-src} lines for \textsl{squeeze} in your \texttt{/etc/apt/sources.list}, you can use:\\
		\texttt{apt-get source grep=2.6.3-3}\\
		or \texttt{apt-get source grep/stable}\\
		or, if you feel lucky: \texttt{apt-get source grep}
	\hbr
	\item The \texttt{grep} source package is composed of three files:
		\begin{itemize}
			\item \texttt{grep\_2.6.3-3.dsc}
			\item \texttt{grep\_2.6.3-3.debian.tar.bz2}
			\item \texttt{grep\_2.6.3.orig.tar.bz2}
		\end{itemize}
		This is typical of the "3.0 (quilt)" format.
	\hbr
\item If needed, uncompress the source with\\
	\texttt{dpkg-source -x grep\_2.6.3-3.dsc}
\end{itemize}
\end{frame}

\begin{frame}{Looking around and building the package}
	\begin{enumerate}
			\setcounter{enumi}{1}
	\item Look at the files in \texttt{debian/}.
		\begin{itemize}
			\item 		How many binary packages are generated by this source package?
			\item 		Which packaging helper does this package use?
		\end{itemize}
	\end{enumerate}
	\hbr
	\begin{itemize}
		\item According to \texttt{debian/control}, this package only generates one binary package, named \texttt{grep}.
			\hbr
		\item According to \texttt{debian/rules}, this package is typical of \textsl{classic} debhelper packaging, without using \textsl{CDBS} or \textsl{dh}. One can see the various calls to \texttt{dh\_*} commands in \texttt{debian/rules}.
	\end{itemize}
	\hbr
	\begin{enumerate}
			\setcounter{enumi}{2}

		\item Build the package
	\end{enumerate}
	\hbr
	\begin{itemize}
		\item Use \texttt{apt-get build-dep grep} to fetch the build-dependencies
		\item Then \texttt{debuild} or \texttt{dpkg-buildpackage -us -uc} (Takes about 1 min)
	\end{itemize}
\end{frame}

\begin{frame}{Editing the changelog}
	\begin{enumerate}
			\setcounter{enumi}{3}

	\item We are now going to modify the package. Add a changelog entry and increase the version number.
	\end{enumerate}
	\hbr
	\begin{itemize}
		\item \texttt{debian/changelog} is a text file. You could edit it and add a new entry manually.
	\hbr
		\item Or you can use \texttt{dch -i}, which will add an entry and open the editor
	\hbr
		\item The name and email can be defined using the \texttt{DEBFULLNAME} and \texttt{DEBEMAIL} environment variables
	\hbr
		\item After that, rebuild the package: a new version of the package is built
	\hbr
		\item Package versioning is detailed in section 5.6.12 of the Debian policy\\
			\url{http://www.debian.org/doc/debian-policy/ch-controlfields.html}
	\end{itemize}
\end{frame}

\begin{frame}{Disabling Perl regexp support and rebuilding}
	\begin{enumerate}
			\setcounter{enumi}{4}

	\item Now disable perl-regexp support (it is a \texttt{./configure} option)
	\item Rebuild the package
	\end{enumerate}
	\hbr
	\begin{itemize}
		\item Check with \texttt{./configure --help}: the option to disable Perl regexp is \texttt{--disable-perl-regexp}
	\hbr
		\item Edit \texttt{debian/rules} and find the \texttt{./configure} line
	\hbr
		\item Add \texttt{--disable-perl-regexp}
	\hbr
		\item Rebuild with \texttt{debuild} or \texttt{dpkg-buildpackage -us -uc}
	\end{itemize}
\end{frame}

\begin{frame}{Comparing and testing the packages}
	\begin{enumerate}
			\setcounter{enumi}{6}

	\item Compare the original and the new package with debdiff
	\item Install the newly built package
	\end{enumerate}
	\hbr
	\begin{itemize}
		\item Compare the binary packages: \texttt{debdiff ../*changes}
	\hbr
		\item Compare the source packages: \texttt{debdiff ../*dsc}
	\hbr
		\item Install the newly built package: \texttt{debi}\\
			Or \texttt{dpkg -i ../grep\_<TAB>}
	\hbr
		\item \texttt{grep -P foo} no longer works!
	\end{itemize}
\br
\begin{enumerate}
		\setcounter{enumi}{8}
	\item Cry if you messed up ;)
\end{enumerate}
\hbr
Or not: reinstall the previous version of the package:
\begin{itemize}
	\item \texttt{apt-get install --reinstall grep=2.6.3-3} \textit{(= previous version)}
\end{itemize}
\end{frame}

\end{document}
