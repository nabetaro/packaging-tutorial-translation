\documentclass[10pt,final]{beamer}
\mode<presentation>
\usetheme{debian}
\usepackage{debiantutorial}

\hypersetup{pdftitle={Practical session 0: dash},bookmarks}
\title[Practical session 0: dash]{Practical session 0:\\ Simple rebuild of the Dash package}
\author[]{Philip Hands\\{\small\texttt{phil@hands.com}}}
\institute{\includegraphics[viewport=274 335 360 440,width=1cm]{figs/openlogo-nd.pdf}}
\date{}

\begin{document}

\frame{\titlepage}

\begin{frame}{Practical session 0: building the dash package}
\begin{enumerate}
\item Install build-essential debhelper devscripts\\
  {\small \texttt{apt-get install build-essential debhelper devscripts}}
  \hbr
\item Create a working directory, and get in it:\\
  \begin{tabbing}
    e.g. \= \texttt{mkdir \textasciitilde/src/\textsl{packagename} ; chdir \textasciitilde/src/\textsl{packagename}}\\
    or   \> \texttt{mkdir \textasciitilde/debian ; chdir \textasciitilde/debian}
  \end{tabbing}
  
\item Grab the \texttt{dash} source package\\
  \texttt{apt-get source dash}\\ 
  {\small this needs you to have \texttt{deb-src} lines in your \texttt{/etc/apt/sources.list}}
\item Build the package\\
  {\small \texttt{cd dash-*\\ debuild -us -uc}}

\item Check that it worked\\
  {\small You should notice that we've got some new
    files in the directory above us, in particular
    there should now be a couple of \texttt{.deb}
    files}
\end{enumerate}
\end{frame}

\end{document}
